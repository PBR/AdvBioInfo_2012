% This template from http://www.vel.co.nz

% LaTeX resume using res.cls
\documentclass[margin]{res}
\setlength{\textwidth}{5.1in} % set width of text portion

\usepackage{hyperref}


\hypersetup{
    colorlinks=true,       % false: boxed links; true: colored links
    urlcolor=red           % color of external links
}


\begin{document}

\moveleft.5\hoffset\centerline{
    \large\bf BIF-30806 - 2012 - Introduction to data integration
}

\moveleft.5\hoffset\centerline{
    \large\bf - \textit{Exercise} -
}

\moveleft\hoffset\vbox{\hrule width\resumewidth height 1pt}\smallskip

\begin{resume}
 
\section{OBJECTIVE}  For the protein with the ID Q42435 retrieve the name,
                short name and pathways from the uniprot website.
 

\section{HTML} Retrieve and extract the asked information from the HTML
            available at : \\
            \url{http://www.uniprot.org/uniprot/Q42435}
 
 
\section{XML} XML is a well-used format. If you start using web-services
            chances are that you will have to face it one day or another.
            Here we are going to parse the XML without calling a web-service
            as it would fall beyond the scope of this exercise.
            
            Retrieve and extract the asked information from the XML 
            available at : \\
            \url{http://www.uniprot.org/uniprot/Q42435.xml}
 
\section{RDF} RDF is one of the format which is used by the semantic web
            to represent triples.
            
            Uniprot provides protein information in RDF.
            
            Retrieve and extract the asked information from the RDF
            available at : \\
            \url{http://www.uniprot.org/uniprot/Q42435.rdf}

\section{Extras-information} This exercise can be made using the programming
                languages perl or python as you prefer, just remember that
                the goal of this course is also to learn perl.

                If you do not know where to start, you can find help starting
                at: \\
                % We need to put them in a different place - Will they have git?
                \url{http://www.bioinformatics.nl/courses/BIF-30806/} \\
                For each format these scripts retrieve the name of the protein,
                you will have to adjust them for the short name and the pathways.
                
                Some more hints:
                \begin{itemize}
                    \item To parse the HTML, you will probably want to use regular expression
                    \item To parse the XML or the RDF there are some modules available to help you (see below).
                \end{itemize}
                If you use perl:
                \begin{itemize}
                    \item To parse the xml, you will want to look at the XML::Simple module \\
                    See: \url{http://search.cpan.org/dist/XML-Simple/lib/XML/Simple.pm}
                    \item To parse the rdf, you will want to look at the RDF::Trine module \\
                    See: \url{http://search.cpan.org/dist/RDF-Trine/lib/RDF/Trine.pm}
                \end{itemize}
                
                If you use python:
                \begin{itemize}
                    \item To parse the xml, you will want to look at the python-lxml module \\
                    See: \url{http://lxml.de} \\
                    Note: In the example file there is a small hack which might help you!
                    \item To parse the rdf, you will want to look at the python-rdflib module \\
                    See: \url{http://code.google.com/p/rdflib/}
                \end{itemize}
                

\end{resume}
\end{document}
